\chapter{Riemann-Stieltjes Integral}
\begin{define}[Partition]
	A partition $P$ of $[a,b]$ is $\{x_0,x_1,x_2,\ldots ,x_{n}\}$ for some $n\ge 1$, with $a=x_0\le x_1 \ldots \le x_{n-1}\le x_{n}=b$.\\
	\begin{notation}
		\begin{describe}
			\item[$\Delta x_i$]$=x_i-x_{i-1}$ for $i=1,\ldots ,n$
			\item[$f$]$: [a,b]\to \R$ be bounded, which is not necessarily continuous
			\item[$M_i$]$=\sup\{f(x):x_{i-1}\le x\le x_i\}$, $m_i=\inf\{f(x):x_{i-1}\le x\le x_i\}$
			\item[$U(P,f)$]$=\sum_{i=1}^{n}{M_i \Delta x_i}$
			\item[$L(P,f)$]$=\sum_{i=1}^{n}{m_{i}\Delta x_{i}}$.
		\end{describe}
	\end{notation}
	\begin{note}
		$L(P,f)\le U(p,f)$ always.
	\end{note}
\end{define}

\begin{define}[Riemann Integral]
	\begin{description}
		\item[Upper Riemann Integral]: $\overline{\int_{a}^{b}}f(x)dx=\inf_{P}\{ U(P,f)\}=\inf\{U(P,f):P\text{ is a partition of }[a,b]\}$.
		\item[Lower Riemann Integral]: $\underline{\int_{a}^{b}}f(x)dx=\sup_{P}\{L(P,f)\} =\sup\{L(P,f):P\text{ is a partition of }[a,b]\}$.
		\item[Riemann Integrable]: $f$ is Riemann integrable on $[a,b]$ if $\overline{\int_{a}^{b}}f(x)dx=\underline{\int_{a}^{b}}f(x)dx$.
		      If $f$ is Riemann integrable on $[a,b]$, we write $f \in \mathscr{R}[a,b]$ and
		      \[
			      \int_{a}^{b}f(x)dx=\overline{\int_{a}^{b}}f(x)dx=\underline{\int_{a}^{b}}f(x)dx.
		      \]

	\end{description}\hfill\\
	\begin{note}
		Since $f$ is bounded, $m=\inf\{f(x):a\le x\le b\}$ and $M=\sup\{f(x):a\le x\le b\}$ are both finite. Hence, for any $P$, $m\le m_i\le M_i\le M$ and $\forall_{i}: m(b-a)\le L(P,f)\le U(P,f)\le M(b-a)$.
	\end{note}
\end{define}
\hfill
\begin{notation}
	Let $\alpha:[a,b]\to \R$ is a monotone increasing function.
	Then $\Delta \alpha_i=\alpha(x_{i})-\alpha(x_{i-1})$.
\end{notation}

\begin{define}[2]
	Given $P$, let $\Delta \alpha_i=\alpha(x_{i})-\alpha(x_{i-1})$. (Note: $\Delta \alpha_i\ge 0$).\\
	For bounded $f$, let $U(P,f,\alpha)=\sum_{i=1}^{n}{M_{i} \Delta \alpha_i}$, $L(P,f,\alpha)=\sum_{i=1}^{n}{m_{i}}\Delta \alpha_i$.\\
	\begin{description}
		\item[Upper Riemann-Stieltjes Integral] $\overline{\int_{a}^{b}}f(x)d\alpha=\overline{\int_{a}^{b}}f(x)d\alpha(x)=\inf_{P}\{U(P,f,\alpha)\}=\inf\{U(P,f,\alpha):P\text{ is a partition of }[a,b]\}$.
		\item[Lower Riemann-Stieltjes Integral] $\underline{\int_{a}^{b}}f(x)d\alpha=\underline{\int_{a}^{b}}f(x)d\alpha(x)=\sup_{P}\{L(P,f,\alpha)\}=\sup\{L(P,f,\alpha):P\text{ is a partition of }[a,b]\}$.
	\end{description}\hfill\\
	If $\overline{\int_{a}^{b}}f(x)d\alpha=\underline{\int_{a}^{b}}f(x)d\alpha$, then $f\in R[a,b,\alpha]$ and $\int_{a}^{b}f(x)d\alpha=\overline{\int_{a}^{b}}f(x)d\alpha=\underline{\int_{a}^{b}}f(x)d\alpha$.\\
	If $\alpha(x)=x$, then equivalent to $\int_{a}^{b}f(x)dx$.
\end{define}

\begin{define}[3]
	\begin{enumerate}
		\item Partition $P^{*}$ is called a refinement of $P$ if $P\subset P^{*}$.
		\item Partition $P^{*}$ is called the common refinement of $P_1$ and $P_2$ if $P^{*}=P_1 \cup P_2$.
	\end{enumerate}
\end{define}

\begin{thm}[4]
	If $P^{*}$ is a refinement of $P$ then $L(P,f,\alpha)\le L(P^{*},f,\alpha)\le U(P^{*},f,\alpha)\le U(P,f,\alpha)$.
	\begin{proof}
		It's enough to consider $p^{*}$ with one extra point: $x_{i-1}\le x^{*}\le x_{i}$.\\
		Sketch for $L$:\\
		\begin{align*}
			 & L(P^{*},f,\alpha)-L(p,f,\alpha)                                                   \\
			 & = m^{*}[\alpha(x^{*})-\alpha(x_{i-1})]+m_i[\alpha(x_i)\alpha(x^{*})]
			-m_{i}[\alpha(x^{*})-\alpha(x_{i-1})]-m_i[\alpha(x_i)-\alpha(x^{*})]                 \\
			 & = (m^{*}-m_i)[\alpha(x^{*})-\alpha(x_{i-1})]+(m_i-m_i)[\alpha(x_i)-\alpha(x^{*})] \\
		\end{align*}
	\end{proof}
\end{thm}

\begin{notation}
	When $f,\alpha$ are fixed, we write $L(P)=L(P,f,\alpha),U(P)=U(P,f,\alpha)$
\end{notation}

\begin{thm}[5]
	$\underline{\int_{a}^{b}{f\mathrm{d}\alpha}}\le  \overline{\int_{a}^{b}{f\mathrm{d}\alpha}}$.
	\begin{proof}
		For partitions $P_1,P_2$, let $P^{*}=P_1 \cup P_2$.
		By Theorem~\ref{thm:6.4}, $L(P_1)\le L(P^{*})\le U(P^{*})\le U(P_2)$.
		In particular, $\sup_{P_1}\{L(P_1)\}\le U(P_2)$ for all $P_2$.
		Hence, $\sup_{P_1}\{L(P_1)\}\le \inf_{P_2}\{U(P_2)\}$.
	\end{proof}
\end{thm}
\begin{thm}[6]
	$f \in \mathscr{R}_{\alpha}[a,b]\Leftrightarrow \forall_{\epsilon > 0}: \exists P_{\epsilon} \text{ s.t. } U(P_{\epsilon})-L(P_{\epsilon})<\epsilon$
	\begin{proof}
		Let $\epsilon>0$.
		\begin{description}
			\item[$(\Rightarrow)$]
			      By hypothesis, $\sup_{P}\{L(P)\}= \underline{\int_{a}^{b}}f\mathrm{d}\alpha=\overline{\int_{a}^{b}}f\mathrm{d}\alpha=\inf_{P}\{U(P)\}$.\\
			      $\exists P_1,P_2$ s.t. $L(P_1)>\underline{\int_{a}^{b}}f\mathrm{d}\alpha-\epsilon/2$ and $U(P_2)<\overline{\int_{a}^{b}}f\mathrm{d}\alpha+\epsilon/2$.\\
			      Then $U(P_{2})-L(P_{1})<\epsilon$. Let $P_{\epsilon}=P^{*}=P_{1}\cup P_{2}$.
			      By Theorem~\ref{thm:6.4}, $L(P_{1})\le L(P^{*})\le U(P^{*})\le U(P_{2})$, so $U(P_{\epsilon})-L(P_{\epsilon})\le U(P_2)-L(P_1)<\epsilon$.
			\item[$(\Leftarrow)$]
			      $0\le \overline{\int_{a}^{b}{f\mathrm{d}\alpha}} - \underline{\int_{a}^{b}{f\mathrm{d}\alpha}}\le U(P_{\epsilon})-L(P_{\epsilon})<\epsilon$. Since $\epsilon$ is arbitrary, $\overline{\int_{a}^{b}{f\mathrm{d}\alpha}} = \underline{\int_{a}^{b}{f\mathrm{d}\alpha}}$.
		\end{description}
	\end{proof}
	\begin{remark}
		very important
	\end{remark}
\end{thm}

\begin{thm}[7]
	Let $\epsilon_0>0$ be fixed. Suppose there exists a partition $P=\{x_0=a,\ldots ,x_n=b\} $ s.t. $U(P,f,\alpha)-L(P,f,\alpha)<\epsilon_0$.
	Let $s_{i},t_{i}$ are arbitrary points in $[x_{i-1},x_i]$.
	Then,
	\begin{enumerate}
		\item For any refinement of $P$, denoted by $P^{*}$, $U(P^{*},f,\alpha)-L(P^{*},f,\alpha)<\epsilon_0$ also holds true
		\item  $\sum_{i=1}^{n}{\left|f(s_i)-f(t_i)\right| \Delta \alpha_i}<\epsilon_0$
		\item If $f \in \mathscr{R}_{\alpha}$, then $\left| \sum_{i=1}^{n}{f(t_i) \Delta \alpha_i}-\int_{a}^{b}{f\mathrm{d}\alpha} \right| <\epsilon_0$
	\end{enumerate}
\end{thm}


\begin{thm}[8]
	If $f$ is continuous on $[a,b]$ then $f \in \mathscr{R}_{\alpha}[a,b]$.
	\begin{proof}
		For any $P$, $U(P)-L(P)=\sum_{i=1}^{n}{(M_i - m_i) \Delta \alpha_i}$.
		Since $[a,b]$ is compact, $f$ is uniformly continuous on $[a,b]$ (Theorem~\ref{thm:4.19}), so $\forall{\eta > 0}: \exists{\delta > 0} \text{ s.t. } |x-t|<\delta \implies |f(x)-f(t)|<\eta$.
		Given $\epsilon > 0$, choose $\eta$ s.t. $\eta[\alpha(b)-\alpha(a)]<\epsilon$ and choose $P$ with $\Delta x_i<\delta=\delta(\eta)$ for all $i$.
		For such $P$, $M_i-m_i \le \eta$.
		Then $U(P)-L(P)\le \sum_{i=1}^{n}{\eta \Delta \alpha_i }=\eta[\alpha(b)-\alpha(a)]<\epsilon$.
		Therefore, $f \in \mathscr{R}_{\alpha}[a,b]$.
	\end{proof}
\end{thm}

\begin{thm}[9]
	If $f$ is monotone increasing or decreasing on $[a,b]$ and $\alpha$ is continuous on $[a,b]$ then $f \in \mathscr{R}_{\alpha}[a,b]$.
	\begin{proof}
		By definition, $U(P)-L(P)=\sum_{i=1}^{n}{(M_{i}-m_{i}) \Delta \alpha_i}$.
		Given $n \in \N$, let $P$ s.t. $\Delta \alpha_i = \dfrac{\alpha(b)-\alpha(a)}{n}$ for all $i$.
		Such $P$ exists by the intermediate value theorem (Theorem~\ref{thm:4.23}) as $\alpha$ is continuous.
		Then, $U(P)-L(P)=\dfrac{\alpha(b)-\alpha(a)}{n} \sum_{i=1}^{n}{M_{i}-m_{i}}$.
		Suppose $f$ is increasing, so $M_i - m_i=f(x_i)-f(x_{i-1})$.
		Then $U(P)-L(P)=\dfrac{\alpha(b)-\alpha(a)}{n}\sum_{i=1}^{n}{f(x_i)-f(x_{i-1})}=\dfrac{\alpha(b)-\alpha(a)}{n}[f(b)-f(a)]$.
		Given $\epsilon>0$, we can choose $n$ (hence $P$) s.t. $U(P)-L(P)<\epsilon$. Therefore, $f \in \mathscr{R}_{\alpha}[a,b]$ by Theorem~\ref{thm:6.6}.
	\end{proof}
	\begin{note}
		We always assume $\alpha$ is monotone.
	\end{note}
\end{thm}

\begin{thm}[10]
	If $f$ is bounded on $[a,b]$ and has only finitely many discontinuities, and $\alpha$ is continuous at each point where $f$ is not, then $f \in \mathscr{R}_{\alpha}[a,b]$.
	\begin{proof}
		We apply Theorem~\ref{thm:6.6}.
		Use $U(P)-L(P)=\sum_{i=1}^{n}{( M_{i}-m_{i} )\Delta \alpha_{i}}$.
		Let $\epsilon>0$ and $E=\{e_1,\ldots ,e_{k}\} $ be the set of points where $f$ is discontinuous.
		$\alpha$ is assumed to be continuous at each $e_i$, which implies $\exists{(u_j,v_j)} \text{ s.t. } u_j<e_j<v_j \text{ and } \alpha(v_j)-\alpha(u_j)<\epsilon$. (Relax inequality to include equality if $e_1=a$, $e_k=b$).\\
		Let $K=[a,b] \cap \left(\bigcup_{j=1}^{k}(u_j,v_j)\right)^{c}$. $K$ is compact.
		$f$ is continuous on $K$, so $f$ is uniformly continuous on $K$ by Theorem~\ref{thm:4.19}.
		Hence, $\exists{\delta>0} \text{ s.t. } \text{ for } s ,t \in K$, $|s-t|<\delta \implies |f(s)-f(t)|<\epsilon$.\\
		Form $P$ to consist of $\{u_1,v_1,\ldots ,u_{k},v_{k}\}$ and additional points in $K$ so that $\Delta x_i < \delta$.
		If $x_i$ is in $K$, then $M_i-m_i<\epsilon$.
		Otherwise, $x_i=u_j$ or $x_i=v_j$ for some $j$, so $\Delta \alpha_i \le \epsilon$. Then
		\begin{align*}
			0\le U(P)-L(P) & =\sum_{i=1}^{n}{(M_{i}-m_{i}) \Delta \alpha_i}                                                                                                                         \\
			               & \le \underbrace{k \cdot 2M \epsilon}_{\text{from intervals in $\bigcup_{j=1}^{k}(u_j,v_j)$}}+\underbrace{\epsilon[\alpha(b)-\alpha(a)]}_{\text{from intervals in $K$}}
			.
		\end{align*}
		As RHS is as small as we want by taking $\epsilon$ small enough, $f \in \mathscr{R}_{\alpha}[a,b]$.
	\end{proof}
	\begin{remark}
		\begin{enumerate}
			\item Theorem~\ref{thm:6.10} implies part of A1.2 but do the problem from first principles. Do not apply Theorem~\ref{thm:6.10} directly.
			\item A1.4 shows what can happen if $f,\alpha$ are discontinuous at the same point.
		\end{enumerate}
	\end{remark}
\end{thm}

\begin{thm}[11]
	If $f \in \mathscr{R}_{\alpha}[a,b], m \le f(x)\le M$ for all $x \in [a,b]$, and $\phi:[m,M]\to \R$ is continuous, then $\phi \circ f \in \mathscr{R}_{\alpha}[a,b]$.
	\begin{proof}
		Let $\epsilon>0$. As $\phi$ continuous on $[m,M]$, $\phi$ is uniformly continuous on $[m,M]$ by Theorem~\ref{thm:4.19}.
		That is, $\exists{\delta < \epsilon} \text{ s.t. } |\phi(s)-\phi(t)|<\epsilon$ if $|s-t|<\delta$ for $s,t \in [m,M]$.\\
		Since $f \in \mathscr{R}_{\alpha}$, there exists a partition $P=\{x_0,x_1,\ldots ,x_n\}$ of $[a,b]$ such that $U(P)-L(P)<\delta^2$.\\
		Let $A=\{i \in \{1,2,\ldots ,n\}: M_i -m_i < \delta\}, B=\{i \in \{1,2,\ldots ,n\}: M_i -m_i \ge  \delta\}$.
		Note $A \cup B=\{1,2,\ldots ,n\}$.
		\\
		Let $M^{*}_i=\sup\{f(x):x_{i-1}\le x\le x_i\}$ and $m^{*}_i=\inf\{f(x):x_{i-1}\le x\le x_i\}$.
		Suppose $i \in A$. Then $M_i-m_i<\delta$. By definition of $\delta$, this implies $|M^{*}_i-m^{*}_i|\le \epsilon$.\\
		Suppose $i \in B$.
		By definition of $P$,
		\[
			U(P,f,\alpha)-L(P,f,\alpha)=\sum_{i \in B}{(M_i-m_i)\Delta \alpha_i}<\delta ^2.
		\]
		As $M_i-m_i\ge \delta$,
		\[
			\delta \sum_{i \in B}{\Delta \alpha_i}\le \sum_{i \in B}{(M_i-m_i)\Delta \alpha_i}<\delta ^2
			.\]
		Hence, $\sum_{i \in B}{\Delta \alpha_i}<\delta$. Then,
		\begin{align*}
			 & \sum_{i \in B}{(M^{*}_i-m^{*}_i)\Delta \alpha_i}                                                                            \\
			 & \le 2 \cdot \underbrace{\sup\{|\phi|\}}_{=\sup\{|\phi(t)|: m\le t\le M\}} \cdot \left(\sum_{i \in B}{\Delta\alpha_i}\right) \\
			 & < 2\cdot \sup\{|\phi|\} \cdot \delta                                                                                        \\
			 & <2 \cdot \sup\{|\phi|\} \cdot \epsilon.
		\end{align*}
		Therefore,
		\begin{align*}
			 & U(P,\phi \circ f,\alpha)-L(P,\phi \circ f,\alpha)=\sum_{i=1}^{n}{(M^{*}_i-m^{*}_i)\Delta \alpha_i} \\
			 & =\sum_{i \in A}{(M^{*}_i-m^{*}_i)\Delta \alpha_i}+\sum_{i \in B}{(M^{*}_i-m^{*}_i)\Delta \alpha_i} \\
			 & < \epsilon[(\alpha(b)-\alpha(a)) + 2 \cdot \sup\{|\phi|\}]
		\end{align*}

	\end{proof}
	\begin{example}
		$f \in \mathscr{R}_{\alpha}[a,b]\implies f^2 \in \mathscr{R}_{\alpha}[a,b], |f| \in \mathscr{R}_{\alpha}[a,b]$ where $\phi(t)=t^2$ and $\phi(t)=|t|$ respectively.
	\end{example}
	\begin{note}
		$\phi \in \mathscr{R}_{\alpha}[m,M]$ does not imply $\phi \circ f \in \mathscr{R}_{\alpha}[a,b]$. See A2.
	\end{note}
\end{thm}

\begin{thm}[12](Linearity and related properties)
	\begin{enumerate}
		\item If $f,f_1,f_2 \in \mathscr{R}_{\alpha}[a,b]$, then $f_1+f_2 \in \mathscr{R}_{\alpha}[a,b]$, $cf \in \mathscr{R}_{\alpha}[a,b]$, and
		      \begin{align*}
			      \int_{a}^{b}{(f_1+f_2)\mathrm{d}\alpha} & =\int_{a}^{b}{f_1\mathrm{d}\alpha}+\int_{a}^{b}{f_2\mathrm{d}\alpha} \\
			      \int_{a}^{b}{c f\mathrm{d}\alpha}       & =c\int_{a}^{b}{f\mathrm{d}\alpha}.
		      \end{align*}
		      \begin{proof}
			      TEXTBOOK
		      \end{proof}
		\item $f_1,f_2 \in \mathscr{R}_{\alpha}$ and $f_1(x)\le f_2(x)$ for all $x \in [a,b]$, then $\int_{a}^{b}{f_1\mathrm{d}\alpha}\le \int_{a}^{b}{f_2\mathrm{d}\alpha}$.
		      \begin{proof}
			      $L(P,f_1)\le L(P,f_2)\le \sup_P L(P,f_2)=\int_{a}^{b}{f_2\mathrm{d}\alpha}$.
			      $\int_{a}^{b}{f_1\mathrm{d}\alpha}=\sup_{P}L(P,f_1)\le  \int_{a}^{b}{f_2\mathrm{d}\alpha}$.
		      \end{proof}
		\item If $f \in \mathscr{R}_{\alpha}[a,b], c \in [a,b]$ then $f \in \mathscr{R}_{\alpha}[a,c] \text{ and } f \in \mathscr{R}_{\alpha}[a,b] \text{ and }  \int_{a}^{b}{f\mathrm{d}\alpha}=\int_{a}^{c}{f\mathrm{d}\alpha}+\int_{c}^{b}{f\mathrm{d}\alpha}$.
		\item If $f \in \mathscr{R}_{\alpha} \text{ and }  |f(x)|\le M$ for all $x \in [a,b]$, then $|\int_{a}^{b}{f\mathrm{d}\alpha}|\le M(\alpha(b)-\alpha(a))$.
		      \begin{proof}
			      Let $P=\{a,b\}$. Then $-M[\alpha(b)-\alpha(a)]\le m_1 \Delta \alpha_1=L(P)\le \int_{a}^{b}{f\mathrm{d}\alpha}\le U(P)=M_1 \Delta \alpha_1\le M [\alpha(b)-\alpha(a)]$.
		      \end{proof}
		\item If $f \in \mathscr{R}_{\alpha_1} \text{ and }  f \in \mathscr{R}_{\alpha_2}$, then $f \in \mathscr{R}_{\alpha_1+\alpha_2}$ and
		      \begin{equation*}
			      \label{eq:s}
			      \int_{a}^{b}{f\mathrm{d}(\alpha_1+\alpha_2)}=\int_{a}^{b}{f\mathrm{d}\alpha_1}+\int_{a}^{b}{f\mathrm{d}\alpha_2} \tag{*}.
		      \end{equation*}\\
		      If $f \in \mathscr{R}_{\alpha} \text{ and }  c\ge 0$ then $f \in \mathscr{R}_{\alpha}$ and
		      $\int_{a}^{b}{f\mathrm{d}c \alpha}= c \int_{a}^{b}{f\mathrm{d}\alpha}$.
		      \begin{proof}[\ref{eq:s}]
			      Let $\epsilon>0$. Choose $P_1,P_2$ s.t. $U(P_{j},f,\alpha_j)-L(P_{j},f,\alpha_j)<\dfrac{\epsilon}{2}$, where $j=1,2$.
			      Let $P^{*}=P_1 \cup P_2$. By Theorem~\ref{thm:6.4},
			      \begin{equation*}
				      U(P^{*},f,\alpha_j)-L(P^{*},f,\alpha_j)<\dfrac{\epsilon}{2} \label{eq:ss}\tag{**}
				      .
			      \end{equation*}
			      Since $(\Delta \alpha_1)_i+(\Delta \alpha_2)_i=(\Delta(\alpha_1+\alpha_2))_i$,
			      \begin{align*}
				       & U(P^{*},f,(\alpha_1+\alpha_2))-L(P^{*},f,(\alpha_1+\alpha_2))                                                                     \\
				       & =\sum_{i=1}^{n}{(M_i-m_i)(\Delta(\alpha_1+\alpha_2))_i}                                                                           \\
				       & =\sum_{i=1}^{n}{(M_i-m_i)[(\Delta \alpha_1)_i+(\Delta \alpha_2)_i]}                                                               \\
				       & =\sum_{i=1}^{n}{(M_i-m_i)(\Delta \alpha_1)_i}+\sum_{i=1}^{n}{(M_i-m_i)(\Delta \alpha_2)_i}                                        \\
				       & =U(P^{*},f,\alpha_1)-L(P^{*},f,\alpha_1)+U(P^{*},f,\alpha_2)-L(P^{*},f,\alpha_2)<\dfrac{\epsilon}{2}+\dfrac{\epsilon}{2}=\epsilon
				      .\end{align*}
			      % adding gives $U(P^{*},f,\alpha_1+\alpha_2)-L(P^{*},f,\alpha_1+\alpha_2)<\dfrac{\epsilon}{2}+\dfrac{\epsilon}{2}=\epsilon$.
			      By Theorem~\ref{thm:6.6}, $f \in \mathscr{R}_{\alpha_1+\alpha_2}$.
			      Also $\int_{a}^{b}{f\mathrm{d}(\alpha_1+\alpha_2)}\le U(P^{*},f,\alpha_1+\alpha_2)=U(P^{*},f,\alpha_1)+U(P^{*},f,\alpha_2)<\int_{a}^{b}{f\mathrm{d}\alpha_1}+\dfrac{\epsilon}{2}+\int_{a}^{b}{f\mathrm{d}\alpha_2}+\dfrac{\epsilon}{2}$ by \eqref{eq:ss}.
			      Similarly, $\int_{a}^{b}{f\mathrm{d}(\alpha_1+\alpha_2)}\ge L(P^{*},f,\alpha_1+\alpha_2)=L(P^{*},f,\alpha_1)+L(P^{*},f,\alpha_2)>\int_{a}^{b}{f\mathrm{d}\alpha_1}-\dfrac{\epsilon}{2}+\int_{a}^{b}{f\mathrm{d}\alpha_2}-\dfrac{\epsilon}{2}$ by \eqref{eq:ss}. As $\epsilon$ is arbitrary, \eqref{eq:s} holds.
		      \end{proof}
	\end{enumerate}
\end{thm}
\begin{thm}[13]
	\begin{enumerate}
		\item $f,g \in \mathscr{R}_{\alpha}\implies fg \in \mathscr{R}_{\alpha}$
		      \begin{proof}
			      By Theorem~\ref{thm:6.11} with $\phi(t)=t^2$, $h \in \mathscr{R}_{\alpha}\implies h^2 \in \mathscr{R}_{\alpha}$.
			      By Theorem~\ref{thm:6.12}(a), $f,g \in \mathscr{R}_{\alpha}\implies f+g \in \mathscr{R}_{\alpha}$, so $(f\pm g)^2 \in \mathscr{R}_{\alpha}$.
			      % so $f^2 \in \mathscr{R}_{\alpha}$.
			      Since $(f+g)^2-(f-g)^2=4fg$, by Theorem~\ref{thm:6.12}(a), $fg \in \mathscr{R}_{\alpha}$.
		      \end{proof}
		\item If $f \in \mathscr{R}_{\alpha}$, then $|f| \in \mathscr{R}_{\alpha}$, and $\left|\int_{a}^{b}{f\mathrm{d}\alpha}\right|\le \int_{a}^{b}{|f|\mathrm{d}\alpha}$.
		      \begin{proof}
			      By Theorem~\ref{thm:6.11}, $|f| \in \mathscr{R}_{\alpha}$ (take $\phi(t)=|t|$).
			      Let \[
				      c=\sgn\left( \int_{a}^{b}{f\mathrm{d}\alpha}\right)=
				      \begin{cases}
					      +1 & \text{if } \int_{a}^{b}{f\mathrm{d}\alpha}>0 \\
					      0  & \text{if } \int_{a}^{b}{f\mathrm{d}\alpha}=0 \\
					      -1 & \text{if } \int_{a}^{b}{f\mathrm{d}\alpha}<0
				      \end{cases}
				      .\]
			      As $cf\le |f|$,
			      $\left|   \int_{a}^{b}{f\mathrm{d}\alpha}\right|=c\int_{a}^{b}{f\mathrm{d}\alpha}=\int_{a}^{b}{cf\mathrm{d}\alpha}\le \int_{a}^{b}{|f|\mathrm{d}\alpha}$.
		      \end{proof}
	\end{enumerate}

\end{thm}
\begin{thm}[15]
	Suppose $f$ is bounded on $[a,b]$ and continuous for $s \in (a,b)$.
	Let \[
		\alpha(x)=\begin{cases}
			0 & x\le s \\
			1 & x>s
		\end{cases}
		.\]
	Then $\int_{a}^{b}{f\mathrm{d}\alpha}$ exists, and
	\[
		\int_{a}^{b}{f\mathrm{d}\alpha}=f (s).
	\]
\end{thm}

\begin{remark}
	\begin{enumerate}
		\item By Theorem~\ref{thm:4.29}, if $\alpha$ is monotone-increasing,
		      then $\alpha(x^{+})$ and $\alpha(x^{-})$ exist for all $x \in (a,b)$, and $\alpha({x^{-}})\le \alpha(x)\le \alpha(x^{+})$.
		\item In Theorem~\ref{thm:6.15}, $\alpha$ is left-continuous at $s$.
		      \begin{exercise}
			      Prove the same conclusion for $\alpha(x)=\begin{cases}
					      0 & x<s    \\
					      1 & x\ge s
				      \end{cases}$.
		      \end{exercise}
	\end{enumerate}
\end{remark}

