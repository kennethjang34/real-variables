\chapter{Riemann-Stieltjes Integral}
\begin{define}[Partition]
	A partition $P$ of $[a,b]$ is $\{x_0,x_1,x_2,\ldots ,x_{n}\}$ for some $n\ge 1$, with $a=x_0\le x_1 \ldots \le x_{n-1}\le x_{n}=b$.\\
	\begin{notation}
		\begin{describe}
			\item[$\Delta x_i$]$=x_i-x_{i-1}$ for $i=1,\ldots ,n$
			\item[$f$]$: [a,b]\to \R$ be bounded, which is not necessarily continuous
			\item[$M_i$]$=\sup\{f(x):x_{i-1}\le x\le x_i\}$, $m_i=\inf\{f(x):x_{i-1}\le x\le x_i\}$
			\item[$U(P,f)$]$=\sum_{i=1}^{n}{M_i \Delta x_i}$
			\item[$L(P,f)$]$=\sum_{i=1}^{n}{m_{i}\Delta x_{i}}$.
		\end{describe}
	\end{notation}
	\begin{note}
		$L(P,f)\le U(p,f)$ always.
	\end{note}
\end{define}

\begin{define}[Riemann Integral]
	\begin{description}
		\item[Upper Riemann Integral]: $\overline{\int_{a}^{b}}f(x)dx=\inf_{P}\{ U(P,f)\}=\inf\{U(P,f):P\text{ is a partition of }[a,b]\}$.
		\item[Lower Riemann Integral]: $\underline{\int_{a}^{b}}f(x)dx=\sup_{P}\{L(P,f)\} =\sup\{L(P,f):P\text{ is a partition of }[a,b]\}$.
		\item[Riemann Integrable]: $f$ is Riemann integrable on $[a,b]$ if $\overline{\int_{a}^{b}}f(x)dx=\underline{\int_{a}^{b}}f(x)dx$.
		      If $f$ is Riemann integrable on $[a,b]$, we write $f \in \mathscr{R}[a,b]$ and
		      \[
			      \int_{a}^{b}f(x)dx=\overline{\int_{a}^{b}}f(x)dx=\underline{\int_{a}^{b}}f(x)dx.
		      \]

	\end{description}\hfill\\
	\begin{note}
		Since $f$ is bounded, $m=\inf\{f(x):a\le x\le b\}$ and $M=\sup\{f(x):a\le x\le b\}$ are both finite. Hence, for any $P$, $m\le m_i\le M_i\le M$ and $\forall_{i}: m(b-a)\le L(P,f)\le U(P,f)\le M(b-a)$.
	\end{note}
\end{define}
\hfill
\begin{notation}
	Let $\alpha:[a,b]\to \R$ is a monotone increasing function.
	Then $\Delta \alpha_i=\alpha(x_{i})-\alpha(x_{i-1})$.
\end{notation}

\begin{define}[2]
	Given $P$, let $\Delta \alpha_i=\alpha(x_{i})-\alpha(x_{i-1})$. (Note: $\Delta \alpha_i\ge 0$).\\
	For bounded $f$, let $U(p,f,\alpha)=\sum_{i=1}^{n}{M_{i} \Delta \alpha_i}$, $L(P,f,\alpha)=\sum_{i=1}^{n}{m_{i}}\delta \alpha_i$.\\
	\begin{description}
		\item[Upper Riemann-Stieltjes Integral] $\overline{\int_{a}^{b}}f(x)d\alpha=\overline{\int_{a}^{b}}f(x)d\alpha(x)=\inf_{P}\{U(P,f,\alpha)\}=\inf\{U(P,f,\alpha):P\text{ is a partition of }[a,b]\}$.
		\item[Lower Riemann-Stieltjes Integral] $\underline{\int_{a}^{b}}f(x)d\alpha=\underline{\int_{a}^{b}}f(x)d\alpha(x)=\sup_{P}\{L(P,f,\alpha)\}=\sup\{L(P,f,\alpha):P\text{ is a partition of }[a,b]\}$.
	\end{description}\hfill\\
	If $\overline{\int_{a}^{b}}f(x)d\alpha=\underline{\int_{a}^{b}}f(x)d\alpha$, then $f\in R[a,b,\alpha]$ and $\int_{a}^{b}f(x)d\alpha=\overline{\int_{a}^{b}}f(x)d\alpha=\underline{\int_{a}^{b}}f(x)d\alpha$.\\
	If $\alpha(x)=x$, then equivalent to $\int_{a}^{b}f(x)dx$.
\end{define}

\begin{define}[3]
	\begin{enumerate}
		\item Partition $P^{*}$ is called a refinement of $P$ if $P\subset P^{*}$.
		\item Partition $P^{*}$ is called the common refinement of $P_1$ and $P_2$ if $P^{*}=P_1 \cup P_2$.
	\end{enumerate}
\end{define}

\begin{thm}[4]
	If $P^{*}$ is a refinement of $P$ then $L(P,f,\alpha)\le L(P^{*},f,\alpha)\le U(P^{*},f,\alpha)\le U(P,f,\alpha)$.
	\begin{proof}
		It's enough to consider $p^{*}$ with one extra point: $x_{i-1}\le x^{*}\le x_{i}$.\\
		Sketch for $L$:\\
		\begin{align*}
			 & L(P^{*},f,\alpha)-L(p,f,\alpha)                                                   \\
			 & = m^{*}[\alpha(x^{*})-\alpha(x_{i-1})]+m_i[\alpha(x_i)\alpha(x^{*})]
			-m_{i}[\alpha(x^{*})-\alpha(x_{i-1})]-m_i[\alpha(x_i)-\alpha(x^{*})]                 \\
			 & = (m^{*}-m_i)[\alpha(x^{*})-\alpha(x_{i-1})]+(m_i-m_i)[\alpha(x_i)-\alpha(x^{*})] \\
		\end{align*}
	\end{proof}
\end{thm}

\begin{notation}
	When $f,\alpha$ are fixed, we write $L(P)=L(P,f,\alpha),U(P)=U(P,f,\alpha)$
\end{notation}

\begin{thm}[5]
	$\underline{\int_{a}^{b}{f\mathrm{d}\alpha}}\le  \overline{\int_{a}^{b}{f\mathrm{d}\alpha}}$.
	\begin{proof}
		For partitions $P_1,P_2$, let $P^{*}=P_1 \cup P_2$.
		By Theorem~\ref{thm:6.4}, $L(P_1)\le L(P^{*})\le U(P^{*})\le U(P_2)$.
		In particular, $\sup_{P_1}\{L(P_1)\}\le U(P_2)$ for all $P_2$.
		Hence, $\sup_{P_1}\{L(P_1)\}\le \inf_{P_2}\{U(P_2)\}$.
	\end{proof}
\end{thm}
\begin{thm}[6]
	$f \in \mathscr{R}_{\alpha}[a,b]\Leftrightarrow \forall_{\epsilon > 0}: \exists P_{\epsilon} \text{ s.t. } U(P_{\epsilon})-L(P_{\epsilon})<\epsilon$
	\begin{proof}
		Let $\epsilon>0$.
		\begin{description}
			\item[$(\Rightarrow)$]
			      By hypothesis, $\sup_{P}\{L(P)\}= \underline{\int_{a}^{b}}f\mathrm{d}\alpha=\overline{\int_{a}^{b}}f\mathrm{d}\alpha=\inf_{P}\{U(P)\}$.\\
			      $\exists P_1,P_2$ s.t. $L(P_1)>\underline{\int_{a}^{b}}f\mathrm{d}\alpha-\epsilon/2$ and $U(P_2)<\overline{\int_{a}^{b}}f\mathrm{d}\alpha+\epsilon/2$.\\
			      Then $U(P_{2})-L(P_{1})<\epsilon$. Let $P_{\epsilon}=P^{*}=P_{1}\cup P_{2}$.
			      By Theorem~\ref{thm:6.4}, $L(P_{1})\le L(P^{*})\le U(P^{*})\le U(P_{2})$, so $U(P_{\epsilon})-L(P_{\epsilon})\le U(P_2)-L(P_1)<\epsilon$.
			\item[$(\Leftarrow)$]
			      $0\le \overline{\int_{a}^{b}{f\mathrm{d}\alpha}} - \underline{\int_{a}^{b}{f\mathrm{d}\alpha}}\le U(P_{\epsilon})-L(P_{\epsilon})<\epsilon$. Since $\epsilon$ is arbitrary, $\overline{\int_{a}^{b}{f\mathrm{d}\alpha}} = \underline{\int_{a}^{b}{f\mathrm{d}\alpha}}$.
		\end{description}
	\end{proof}
	\begin{remark}
		very important
	\end{remark}
\end{thm}

\begin{thm}[7]
	Let $\epsilon_0>0$ be fixed. Suppose there exists a partition $P=\{x_0=a,\ldots ,x_n=b\} $ s.t. $U(P,f,\alpha)-L(P,f,\alpha)<\epsilon_0$.
	Let $s_{i},t_{i}$ are arbitrary points in $[x_{i-1},x_i]$.
	Then,
	\begin{enumerate}
		\item For any refinement of $P$, denoted by $P^{*}$, $U(P^{*},f,\alpha)-L(P^{*},f,\alpha)<\epsilon_0$ also holds true
		\item  $\sum_{i=1}^{n}{\left|f(s_i)-f(t_i)\right| \Delta \alpha_i}<\epsilon_0$
		\item If $f \in \mathscr{R}_{\alpha}$, then $\left| \sum_{i=1}^{n}{f(t_i) \Delta \alpha_i}-\int_{a}^{b}{f\mathrm{d}\alpha} \right| <\epsilon_0$
	\end{enumerate}
\end{thm}


\begin{thm}[8]
	If $f$ is continuous on $[a,b]$ then $f \in \mathscr{R}_{\alpha}[a,b]$.
	\begin{proof}
		For any $P$, $U(P)-L(P)=\sum_{i=1}^{n}{(M_i - m_i) \Delta \alpha_i}$
		Since $[a,b]$ is compact, $f$ is uniformly continuous on $[a,b]$ (Theorem~\ref{thm:4.19}), so $\forall{\eta > 0}: \exists{\delta > 0} \text{ s.t. } |x-t|<\delta \implies |f(x)-f(t)|<\eta$.
		Given $\epsilon > 0$, choose $\eta$ s.t. $\eta[\alpha(b)-\alpha(a)]<\epsilon$ and choose $P$ with $\Delta x_i<\delta=\delta(\eta)$ for all $i$.
		Then $U(P)-L(P)<\sum_{i=1}^{n}{\eta \Delta \alpha_i }=\eta[\alpha(b)-\alpha(a)]<\epsilon$.
		Therefore, $f \in \mathscr{R}_{\alpha}[a,b]$.
	\end{proof}
\end{thm}

\begin{thm}[9]
	If $f$ is monotone increasing or decreasing on $[a,b]$ and $\alpha$ is continuous on $[a,b]$ then $f \in \mathscr{R}_{\alpha}[a,b]$.
	\begin{proof}
		By definition, $U(P)-L(P)=\sum_{i=1}^{n}{(M_{i}-m_{i}) \Delta \alpha_i}$.
		Given $n \in \N$, let $P$ s.t. $\Delta \alpha_i = \dfrac{\alpha(b)-\alpha(a)}{n}$ for all $i$.
		Then, $U(P)-L(P)=\dfrac{\alpha(b)-\alpha(a)}{n} \sum_{i=1}^{n}{M_{i}-m_{i}}$.
		Suppose $f$ is increasing, so $M_i - m_i=f(x_i)-f(x_{i-1})$.
		Then $U(P)-L(P)=\dfrac{\alpha(b)-\alpha(a)}{n}\sum_{i=1}^{n}{f(x_i)-f(x_{i-1})}=\dfrac{\alpha(b)-\alpha(a)}{n}[f(b)-f(a)]$.
		Given $\epsilon>0$, we can choose $n$ (hence $P$) s.t. $U(P)-L(P)<\epsilon$. Therefore, $f \in \mathscr{R}_{\alpha}[a,b]$ by Theorem~\ref{thm:6.6}.
	\end{proof}
	\begin{note}
		We always assume $\alpha$ is monotone.
	\end{note}
\end{thm}

\begin{thm}[10]
	If $f$ is bounded on $[a,b]$ and has only finitely many discontinuities, and $\alpha$ is continuous at each point where $f$ is not, then $f \in \mathscr{R}_{\alpha}[a,b]$.
	\begin{proof}
		We apply Theorem~\ref{thm:6.6}.
		Use $U(P)-L(P)=\sum_{i=1}^{n}{( M_{i}-m_{i} )\Delta_{i}}$.
		Let $\epsilon>0$ and $E=\{e_1,\ldots ,e_{k}\} $ be the set of points where $f$ is discontinuous.
		$\alpha$ is assumed to be continuous at each $e_i$, which implies $\exists{(u_j,v_j)} \text{ s.t. } u_j<e_j<v_j \text{ and } \alpha(v_j)-\alpha(u_j)<\epsilon$. (Relax inequality to include equality if $e_1=a$, $e_k=b$)
		Let $K=[a,b] \cap \left(\bigcup_{j=1}^{k}(u_j,v_j)\right)^{c}$. $K$ is compact.
		$f$ is continuous on $K$, so $f$ is uniformly continuous on $K$ by Theorem~\ref{thm:4.19}.
		Hence, $\exists{\delta>0} \text{ s.t. } s , t \in K \text{ and }  |s-t|<\delta \implies |f(s)-f(t)|<\epsilon$.
		Form $P$ to consist of $\{u_1,v_1,\ldots ,u_{k},v_{k}\}$ and additional points $x_i$ in $K$ with $\Delta x_i < \delta$. For such $i$, $M_i-m_i<\epsilon$.
		Then $U(P)-L(P)<\epsilon$.
		For $[u_{j},v_{j}], M_j-m_j \le 2M$, where $M=\sup\{|f(x)|: x \in [a,b]\}$, and $\Delta \alpha_{j}<\epsilon$.
		Then $0\le U(P)-L(P)=\sum_{i=1}^{n}{(M_{i}-m_{i}) \Delta \alpha_i}\le \underbrace{K \cdot 2M \epsilon}_{\text{From $[u_j,v_j]$ intervals}}+\underbrace{\epsilon[\alpha(b)-\alpha(a)]}_{\text{From $K$ intervals}}$.
		RHS is as small as we want by taking $\epsilon$ small enough.
	\end{proof}
	\begin{remark}
		\begin{enumerate}
			\item Theorem~\ref{thm:6.10} implies part of A1.2 but do the problem from first principles. Do not apply Theorem~\ref{thm:6.10} directly.
			\item A1.4 shows what can happen if $f,\alpha$ are discontinuous at the same point.
		\end{enumerate}
	\end{remark}
\end{thm}

\begin{thm}[11]
	If $f \in \mathscr{R}_{\alpha}[a,b], m \le f(x)\le M$ for all $x \in [a,b]$, and $\phi:[m,M]\to \R$ is continuous, then $\phi \circ f \in \mathscr{R}_{\alpha}[a,b]$.
	\begin{proof}
		See textbook. pf similar to pf of Theorem~\ref{thm:6.10}.
	\end{proof}
	\begin{example}
		$f \in \mathscr{R}_{\alpha}[a,b]\implies f^2 \in \mathscr{R}_{\alpha}[a,b], |f| \in \mathscr{R}_{\alpha}[a,b]$ where $phi(t)=t^2$ and $\phi(t)=|t|$ respectively.
	\end{example}
	\begin{note}
		$\phi \in \mathscr{R}_{\alpha}[m,M]$ does not imply $\phi \circ f \in \mathscr{R}_{\alpha}[a,b]$. See A2.
	\end{note}
\end{thm}
