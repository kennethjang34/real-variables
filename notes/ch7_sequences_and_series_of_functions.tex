\chapter{Sequences and Series of Functions}
\begin{example}
	\label{eg:7.1}
	Bad behaviour of limits
	\begin{enumerate}[label=(\arabic*)]
		\item For $m,n \in \N$, let $p_{m,n}=\frac{m}{n}$.
		      Then
		      \[
			      \lim_{m\to \infty}{p_{m,n}}=\infty
		      \]
		      and
		      \[
			      \lim_{n\to \infty}{p_{m,n}}=0
			      .\]
		      Hence,
		      \[
			      \lim_{n\to \infty}{\underbrace{\lim_{m\to \infty}{p_{m,n}}}_{=\infty}}=\infty \neq \lim_{m\to \infty}{\underbrace{\lim_{n\to \infty}{p_{m,n}}}_{=0}}
			      .\]
		      This shows the order of limits matters.
		\item Let
		      \[
			      f_n(x)=
			      \begin{cases}
				      1    & x\ge 0            \\
				      1+nx & -\frac{1}{n}<x<0  \\
				      0    & x\le -\frac{1}{n}
			      \end{cases}
			      .\]
		      Then $f_n(x)$ is a continuous function.
		      Then $\lim_{n\to \infty}{f_n(x)}= \begin{cases}
				      1 & x\ge 0 \\
				      0 & x<0
			      \end{cases}$, which is not continuous at $0$.
		      This shows that the limit of continuous functions need not be continuous.
		      Moreover,
		      \[
			      \lim_{n\to \infty}{\underbrace{\lim_{x\to 0}{f_{n}(x)}}_{=1}}=1
			      ,\]
		      while
		      \[
			      \lim_{x\to 0}{\underbrace{\lim_{n\to \infty}{f_{n}(x)}}_{=f(x)}}
		      \]
		      does not exist. This shows again that the order of limits matters.
		\item
		      For $x \in [0,1]$, let \[
			      f_n(x)= \begin{cases}
				      1 & n!\cdot x \in \Z    \\
				      0 & n!\cdot x \notin \Z \\
			      \end{cases}
			      .\]
		      Each $f_n \in \mathscr{R}[0,1]$ by Theorem~\ref{thm:6.10}.
		      However, \[
			      f(x)=\lim_{n\to \infty}{f_n(x)}= \begin{cases}
				      1 & x \in Q \cap [0,1]    \\
				      0 & x \notin Q \cap [0,1]
			      \end{cases}
			      .\]
		      Hence, $f(x)$ is nowhere continuous, and $f \notin \mathscr{R}[0,1]$.
		\item
		      Let \[
			      f_n(x)= \begin{cases}
				      0       & \left|x\right|\ge \frac{1}{n} \\
				      n^2x +n & -\frac{1}{n}<x<0              \\
				      -n^2x+n & 0<x<\frac{1}{n}               \\
				      0       & x=0
			      \end{cases}
			      .\]
		      Ten $f(x)=\lim_{n\to \infty}{f_n(x)}=0$ for all $x \in \R$.
		      Moreover,
		      \[
			      \forall{n}: \int_{-1}^{1}{f_n(x)\mathrm{d}x}=1
			      ,\]
		      \[
			      \int_{-1}^{1}{f(x)\mathrm{d}x}=0
			      .\]
		      Hence, $\lim_{n\to \infty}{\int_{-1}^{1}{f_n(x)\mathrm{d}x}}=1\neq  \int_{-1}^{1}{\left(\lim_{n\to \infty}{f_n(x)}\right) \mathrm{d}x}$
		\item Let \[
			      f_n(x)=\frac{\sin{nx}}{\sqrt{n}} \text{ for }  n \in \N, x \in \R
			      .\]
		      Let
		      \[
			      f(x)=\lim_{n\to \infty}{f_n(x)}=0
			      ,\]
		      so
		      \[
			      \forall{x \in \R}: f'(x)=0
			      .\]
		      However, \[
			      f'_n(x)= \frac{n \cos{x}}{\sqrt{n}}=\sqrt{n} \cos{nx}
			      .\]
		      Therefore, $f_{n}'(\pi)=\sqrt{n}(-1)^{n}$ diverges as $n\to \infty$. Hence,
		      \[
			      \underbrace{f'(\pi)=\left(\lim_{n\to \infty}{f_n}\right)^{'}(\pi)}_{=0}\neq \underbrace{\lim_{n\to \infty}{f_n'(\pi)}}_{\text{DNE}}
			      .\]
	\end{enumerate}
	These examples show bad behaviour under interchange of limits, which suggests the need of a stronger notion of convergence.
\end{example}

\begin{define}[7]
	Let $E$ be any set and $f_n: E\to \R (\text{ or } \C) $ for $n \in \N$. Then $f_{n}$ converges uniformly to $f$ on $E$ if \[
		\forall{\epsilon>0}: \exists{N} \text{ s.t. } \forall{n\ge N}: \forall{x \in E}: \left|f_{n}(x)-f(x)\right|<\epsilon
		.\]
\end{define}

\begin{example}
	\begin{enumerate}
		\item
		      Consider the example~\ref{eg:7.1}(2)
		      \[
			      f_n(x)=
			      \begin{cases}
				      1    & x\ge 0            \\
				      1+nx & -\frac{1}{n}<x<0  \\
				      0    & x\le -\frac{1}{n}
			      \end{cases}
			      ,\]
		      \[
			      f_n(x)-f(x)=	\begin{cases}
				      1+nx & -\frac{1}{n}<x<0 \\
				      0    & \text{otherwise}
			      \end{cases}
			      .\]
		      In particular, $f_n(-\frac{1}{2n})-f(-\frac{1}{2n})=\frac{1}{2}$, so we cannot choose $N$ s.t. $n\ge N\implies \left|f_n(x)-f(x)\right|<\epsilon=\frac{1}{4}$ for all $x$.
		      \[
			      \therefore f_n \text{ does not converge uniformly to } f \text{ on } \R
			      .\]
		\item
		      Consider the example~\ref{eg:7.1}(5).
		      \[
			      f(x)=\lim_{n\to \infty}{f_n(x)}=0
			      ,\]
		      \[
			      \forall{x \in \R}: f'(x)=0
			      .\]
		      Then \[
			      \left|f_n(x)-\underbrace{f(x)}_{=0}\right|=\left|\frac{\sin{nx}}{\sqrt{n}}\right|\le \frac{1}{\sqrt{n}}
			      ,\]
		      so $f_n\to f$ uniformly on $\R$.
			      [Note: uniform convergence is not enough for $\lim_{n\to \infty}{f_n'}=(\lim_{n\to \infty}{f_n})'$.]
	\end{enumerate}
\end{example}

\begin{thm}[8][Cauchy Criteria for Uniform Convergence]\\
	$f_n$ converges uniformly to $f$ on $E$ if and only if \[
		\forall{\epsilon>0}: \exists{N} \text{ s.t. } m,n\ge \N \implies \forall{x \in E}: \left|f_{n}(x)-f_{m}(x)\right|<\epsilon
		.\]
	That is, we can choose such $N$ independent of $x$.
	\begin{proof}
		\begin{description}
			\item[$(\implies)$]
			      Suppose $f_n$ converges uniformly to $f$ on $E$.
			      Then \[
				      \left|f_n(x)-f_m(x)\right|\le \left|f_n(x)-f(x)\right|+\left|f_m(x)-f(x)\right|
				      .\]
			      For $\epsilon>0$, choose $N$ s.t. $\forall{n\ge N}: \forall{x \in E}: \left|f_{n}(x)-f(x)\right|<\frac{\epsilon}{2}$.
			      Then \[
				      \forall{m,n\ge N}: \forall{x \in E}: \left|f_{n}(x)-f_{m}(x)\right|\le \left|f_{n}(x)-f(x)\right|+\left|f_{m}(x)-f(x)\right|<\epsilon
				      .\]
			\item[$(\impliedby)$]
			      Let $x \in E$. $\{f_{n}(x)\}_{n \in \N}$ is a Cauchy sequence in $\C$, so has a limit $f(x)= \lim_{n\to \infty}{f_{n}(x)}$.
			      To check uniformity, let $\epsilon>0$. We know that $\exists{N} \text{ s.t. } \left|f_{n}(x)-f_{m}(x)\right|<\epsilon$ if $n,m \ge N$ for all $x \in E$.
			      Let $m\to \infty$. Then $\left|f_{n}(x)-f_{m}(x)\right|\le \epsilon$ if $n\ge N$ for all $x \in E$.
		\end{description}
	\end{proof}
\end{thm}


\begin{define}
	$\sum_{n=1}^{\infty}{f_{n}(x)}$ converges uniformly on $E$ if $s_n(x)=\sum_{i=1}^{n}{f_{i}(x)}$	is a uniformly convergence sequence of functions.
\end{define}


\begin{thm}[10][Weierstass M test]
	If $\left|f_{n}(x)\right| \le M_{n}$ for all $n \ge N_0$ and all $x \in E$ and if $\sum_{n=N_0}^{\infty}{M_n} < \infty$, then $\sum_{n=1}^{\infty}{f_{n}(x)}$ converges uniformly on $E$.
	\begin{proof}
		Let $s_{n}(x)=\sum_{i=1}^{n}{f_{i}(x)}$.
		For $n>m\ge N_0$, \[
			\forall{x \in E}:	\left|s_{n}(x)-s_{m}(x)\right| =\left|\sum_{i=m+1}^{n}{f_{i}(x)}\right| \le \sum_{i=m+1}^{n}{M_i}
			.\]
		Let $\epsilon>0$. Choose $N\ge N_0$ s.t. $\sum_{i=N+1}^{\infty}{M_i}<\epsilon$.
		Then $\left|s_{n}(x)-s_{m}(x)\right| <\epsilon$ if $n>m\ge N$ for all $x \in E$. Hence, $s_{n}$ converges uniformly on $E$ by Theorem~\ref{thm:7.8}.
	\end{proof}
\end{thm}

\begin{thm}[11]
	Let $E \subset X$ and $f_{n}: E \to \R (\text{ or } \C)$, $n \in \N$.
	Suppose $f_{n}\to f$ uniformly on $E$. Let $x \in E'$ and suppose $\lim_{t\to x}{f_{n}(t)}=A_{n}$ exists for each $n$.
	Then $A_{n}\to A$ for some $A$ and $\lim_{t\to x}{f(t)}=A$; i.e.,
	\[
		\lim_{t\to x}{\underbrace{\lim_{n\to \infty}{f_{n}(t)}}_{f(t)}}=\lim_{n\to \infty}{\underbrace{\lim_{t\to x}{f_{n}(t)}}_{A_n}}
		.\]
	\begin{proof}
		\begin{describe}
			\item[$A_n \to A$ for some A:]
			It suffices to show that $\{A_{n}\}$ is a Cauchy sequence.
			Given $\epsilon>0$, choose $N$ s.t. $m,n\ge N \implies \left|f_{m}(t)-f_{n}(t)\right| < \epsilon$ for all $t \in E$.
			Let $t\to x$. $\left|A_{m}-A_{n}\right| < \epsilon$ if $m,n\ge N$. Therefore, $\{A_{n}\}$ is a Cauchy sequence.
			\item[$f(t)\to A$ as $t\to x$:]
			For $t \in E$ and $n \in \N$,
			\begin{equation*}
				\left|f(t)-A\right| \le \left|f(t)-f_{n}(t)\right| + \left|f_{n}(t)-A_n\right|  + \left|A_n -A\right| \tag{*}
			\end{equation*}
			Let $\epsilon>0$. Since $f_{n}\to f$ uniformly $\exists{N_1} \text{ s.t. } \left|f(t)-f_{n}(t)\right| <\frac{\epsilon}{3}$ if $n\ge N$ for all $t \in E$.
			Since $A_{n}\to A$, $\exists{N_2} \text{ s.t. } \left|A_{n}-A\right| < \frac{\epsilon}{3}$ if $n\ge N_2$.
			Let $N=\max\{N_1,N_2\}$ and use $n=N$ in (*).
			Then
			\[
				\forall{t \in E}: \left|f(t)-A\right| \frac{\epsilon}{3}+\left|f_N(t)-A_N\right| +\frac{\epsilon}{3}
				.\]
			Since $\lim_{t\to x}{f_N(t)}=A_N$, we can choose $\delta>0$ s.t. $t \in N_{\delta}^{E}(x)\setminus \{x\}  \implies \left|f_N(t)-A_{N}\right| < \frac{\epsilon}{3}$.
			Then $t \in N_{\delta}^{E}(x)\setminus \{x\}  \implies \left|f(t)-A\right|< \frac{\epsilon}{3}+\frac{\epsilon}{3}+\frac{\epsilon}{3}=\epsilon$.
		\end{describe}
	\end{proof}
\end{thm}
\begin{Corollary}
	If $f_{n}$ is continuous on $E$ and $f_{n}\to f$ uniformly on $E$, then $f$ is continuous on $E$.
	\begin{proof}
		Every function is continuous at an isolated point, so only need to consider $x \in E' \cap E$.\\
		Then by Theorem~\ref{thm:7.11},
		\begin{align*}
			f(x) & =\lim_{n\to \infty}{f_{n}(x)}  = \lim_{n\to \infty}{\lim_{t\to x}{f_{n}(t)}} \;\;\;\hfill(\because f_{n} \text{ continuous}) \\
			     & =\lim_{t\to x}{\lim_{n\to \infty}{f_{n}(t)}}=\lim_{t\to x}{f(t)}
			.\end{align*}
	\end{proof}

	\begin{remark}
		VERY IMPORTANT
	\end{remark}
\end{Corollary}

