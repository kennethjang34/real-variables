\chapter{Sequences and Series of Functions}
\begin{example}
	\label{eg:7.1}
	Bad behaviour of limits
	\begin{enumerate}[label=(\arabic*)]
		\item For $m,n \in \N$, let $p_{m,n}=\frac{m}{n}$.
		      Then
		      \[
			      \lim_{m\to \infty}{p_{m,n}}=\infty
		      \]
		      and
		      \[
			      \lim_{n\to \infty}{p_{m,n}}=0
			      .\]
		      Hence,
		      \[
			      \lim_{n\to \infty}{\underbrace{\lim_{m\to \infty}{p_{m,n}}}_{=\infty}}=\infty \neq \lim_{m\to \infty}{\underbrace{\lim_{n\to \infty}{p_{m,n}}}_{=0}}
			      .\]
		      This shows the order of limits matters.
		\item Let
		      \[
			      f_n(x)=
			      \begin{cases}
				      1    & x\ge 0            \\
				      1+nx & -\frac{1}{n}<x<0  \\
				      0    & x\le -\frac{1}{n}
			      \end{cases}
			      .\]
		      Then $f_n(x)$ is a continuous function.
		      Then $\lim_{n\to \infty}{f_n(x)}= \begin{cases}
				      1 & x\ge 0 \\
				      0 & x<0
			      \end{cases}$, which is not continuous at $0$.
		      This shows that the limit of continuous functions need not be continuous.
		      Moreover,
		      \[
			      \lim_{n\to \infty}{\underbrace{\lim_{x\to 0}{f_{n}(x)}}_{=1}}=1
			      ,\]
		      while
		      \[
			      \lim_{x\to 0}{\underbrace{\lim_{n\to \infty}{f_{n}(x)}}_{=f(x)}}
		      \]
		      does not exist. This shows again that the order of limits matters.
		\item
		      For $x \in [0,1]$, let \[
			      f_n(x)= \begin{cases}
				      1 & n!\cdot x \in \Z    \\
				      0 & n!\cdot x \notin \Z \\
			      \end{cases}
			      .\]
		      Each $f_n \in \mathscr{R}[0,1]$ by Theorem~\ref{thm:6.10}.
		      However, \[
			      f(x)=\lim_{n\to \infty}{f_n(x)}= \begin{cases}
				      1 & x \in Q \cap [0,1]    \\
				      0 & x \notin Q \cap [0,1]
			      \end{cases}
			      .\]
		      Hence, $f(x)$ is nowhere continuous, and $f \notin \mathscr{R}[0,1]$.
		\item
		      Let \[
			      f_n(x)= \begin{cases}
				      0       & \left|x\right|\ge \frac{1}{n} \\
				      n^2x +n & -\frac{1}{n}<x<0              \\
				      -n^2x+n & 0<x<\frac{1}{n}               \\
				      0       & x=0
			      \end{cases}
			      .\]
		      Ten $f(x)=\lim_{n\to \infty}{f_n(x)}=0$ for all $x \in \R$.
		      Moreover,
		      \[
			      \forall{n}: \int_{-1}^{1}{f_n(x)\mathrm{d}x}=1
			      ,\]
		      \[
			      \int_{-1}^{1}{f(x)\mathrm{d}x}=0
			      .\]
		      Hence, $\lim_{n\to \infty}{\int_{-1}^{1}{f_n(x)\mathrm{d}x}}=1\neq  \int_{-1}^{1}{\left(\lim_{n\to \infty}{f_n(x)}\right) \mathrm{d}x}$
		\item Let \[
			      f_n(x)=\frac{\sin{nx}}{\sqrt{n}} \text{ for }  n \in \N, x \in \R
			      .\]
		      Let
		      \[
			      f(x)=\lim_{n\to \infty}{f_n(x)}=0
			      ,\]
		      so
		      \[
			      \forall{x \in \R}: f'(x)=0
			      .\]
		      However, \[
			      f'_n(x)= \frac{n \cos{x}}{\sqrt{n}}=\sqrt{n} \cos{nx}
			      .\]
		      Therefore, $f_{n}'(\pi)=\sqrt{n}(-1)^{n}$ diverges as $n\to \infty$. Hence,
		      \[
			      \underbrace{f'(\pi)=\left(\lim_{n\to \infty}{f_n}\right)^{'}(\pi)}_{=0}\neq \underbrace{\lim_{n\to \infty}{f_n'(\pi)}}_{\text{DNE}}
			      .\]
	\end{enumerate}
	These examples show bad behaviour under interchange of limits, which suggests the need of a stronger notion of convergence.
\end{example}

\begin{define}[7]
	Let $E$ be any set and $f_n: E\to \R (\text{ or } \C) $ for $n \in \N$. Then $f_{n}$ converges uniformly to $f$ on $E$ if \[
		\forall{\epsilon>0}: \exists{N} \text{ s.t. } \forall{n\ge N}: \forall{x \in E}: \left|f_{n}(x)-f(x)\right|<\epsilon
		.\]
\end{define}

\begin{example}
	\begin{enumerate}
		\item
		      Consider the example~\ref{eg:7.1}(2)
		      \[
			      f_n(x)=
			      \begin{cases}
				      1    & x\ge 0            \\
				      1+nx & -\frac{1}{n}<x<0  \\
				      0    & x\le -\frac{1}{n}
			      \end{cases}
			      ,\]
		      \[
			      f_n(x)-f(x)=	\begin{cases}
				      1+nx & -\frac{1}{n}<x<0 \\
				      0    & \text{otherwise}
			      \end{cases}
			      .\]
		      In particular, $f_n(-\frac{1}{2n})-f(-\frac{1}{2n})=\frac{1}{2}$, so we cannot choose $N$ s.t. $n\ge N\implies \left|f_n(x)-f(x)\right|<\epsilon=\frac{1}{4}$ for all $x$.
		      \[
			      \therefore f_n \text{ does not converge uniformly to } f \text{ on } \R
			      .\]
		\item
		      Consider the example~\ref{eg:7.1}(5).
		      \[
			      f(x)=\lim_{n\to \infty}{f_n(x)}=0
			      ,\]
		      \[
			      \forall{x \in \R}: f'(x)=0
			      .\]
		      Then \[
			      \left|f_n(x)-\underbrace{f(x)}_{=0}\right|=\left|\frac{\sin{nx}}{\sqrt{n}}\right|\le \frac{1}{\sqrt{n}}
			      ,\]
		      so $f_n\to f$ uniformly on $\R$.
			      [Note: uniform convergence is not enough for $\lim_{n\to \infty}{f_n'}=(\lim_{n\to \infty}{f_n})'$.]
	\end{enumerate}
\end{example}

\begin{thm}[8][Cauchy Criteria for Uniform Convergence]\\
	$f_n$ converges uniformly to $f$ on $E$ if and only if \[
		\forall{\epsilon>0}: \exists{N} \text{ s.t. } m,n\ge \N \implies \forall{x \in E}: \left|f_{n}(x)-f_{m}(x)\right|<\epsilon
		.\]
	That is, we can choose such $N$ independent of $x$.
	\begin{proof}
		\begin{description}
			\item[$(\implies)$]
			      Suppose $f_n$ converges uniformly to $f$ on $E$.
			      Then \[
				      \left|f_n(x)-f_m(x)\right|\le \left|f_n(x)-f(x)\right|+\left|f_m(x)-f(x)\right|
				      .\]
			      For $\epsilon>0$, choose $N$ s.t. $\forall{n\ge N}: \forall{x \in E}: \left|f_{n}(x)-f(x)\right|<\frac{\epsilon}{2}$.
			      Then \[
				      \forall{m,n\ge N}: \forall{x \in E}: \left|f_{n}(x)-f_{m}(x)\right|\le \left|f_{n}(x)-f(x)\right|+\left|f_{m}(x)-f(x)\right|<\epsilon
				      .\]
			\item[$(\impliedby)$]
			      Let $x \in E$. $\{f_{n}(x)\}_{n \in \N}$ is a Cauchy sequence in $\C$, so has a limit $f(x)= \lim_{n\to \infty}{f_{n}(x)}$.
			      To check uniformity, let $\epsilon>0$. We know that $\exists{N} \text{ s.t. } \left|f_{n}(x)-f_{m}(x)\right|<\epsilon$ if $n,m \ge N$ for all $x \in E$.
			      Let $m\to \infty$. Then $\left|f_{n}(x)-f(x)\right| \le \epsilon$ if $n\ge N$ for all $x \in E$.
		\end{description}
	\end{proof}
\end{thm}


\begin{define}
	$\sum_{n=1}^{\infty}{f_{n}(x)}$ converges uniformly on $E$ if $s_n(x)=\sum_{i=1}^{n}{f_{i}(x)}$	is a uniformly convergence sequence of functions.
\end{define}


\begin{thm}[10][Weierstass M test]
	If $\left|f_{n}(x)\right| \le M_{n}$ for all $n \ge N_0$ and all $x \in E$ and if $\sum_{n=N_0}^{\infty}{M_n} < \infty$, then $\sum_{n=1}^{\infty}{f_{n}(x)}$ converges uniformly on $E$.
	\begin{proof}
		Let $s_{n}(x)=\sum_{i=1}^{n}{f_{i}(x)}$.
		For $n>m\ge N_0$, \[
			\forall{x \in E}:	\left|s_{n}(x)-s_{m}(x)\right| =\left|\sum_{i=m+1}^{n}{f_{i}(x)}\right| \le \sum_{i=m+1}^{n}{M_i}
			.\]
		Let $\epsilon>0$. Choose $N\ge N_0$ s.t. $\sum_{i=N+1}^{\infty}{M_i}<\epsilon$.
		Then $\left|s_{n}(x)-s_{m}(x)\right| <\epsilon$ if $n>m\ge N$ for all $x \in E$. Hence, $s_{n}$ converges uniformly on $E$ by Theorem~\ref{thm:7.8}.
	\end{proof}
\end{thm}

\begin{thm}[11]
	Let $E \subset X$ and $f_{n}: E \to \R (\text{ or } \C)$, $n \in \N$.
	Suppose $f_{n}\to f$ uniformly on $E$. Let $x \in E'$ and suppose $\lim_{t\to x}{f_{n}(t)}=A_{n}$ exists for each $n$.
	Then $A_{n}\to A$ for some $A$ and $\lim_{t\to x}{f(t)}=A$; i.e.,
	\[
		\lim_{t\to x}{\underbrace{\lim_{n\to \infty}{f_{n}(t)}}_{f(t)}}=\lim_{n\to \infty}{\underbrace{\lim_{t\to x}{f_{n}(t)}}_{A_n}}=A
		.\]
	\begin{proof}
		\begin{describe}
			\item[$A_n \to A$ for some A:]
			It suffices to show that $\{A_{n}\}$ is a Cauchy sequence.
			Since $f_n\to f$ uniformly on $E$, for any $\epsilon>0$, we can choose $N$ s.t. $m,n\ge N \implies \left|f_{m}(t)-f_{n}(t)\right| < \epsilon$ for all $t \in E$.
			Let $t\to x$. $\left|A_{m}-A_{n}\right| < \epsilon$ if $m,n\ge N$. Therefore, $\{A_{n}\}$ is a Cauchy sequence and converges to some $A$ by completeness of $\R(\text{ or } \C)$.
			\item[$f(t)\to A$ as $t\to x$:]
			For $t \in E$ and $n \in \N$,
			\begin{equation*}
				\left|f(t)-A\right| \le \left|f(t)-f_{n}(t)\right| + \left|f_{n}(t)-A_n\right|  + \left|A_n -A\right| \tag{*}
			\end{equation*}
			Let $\epsilon>0$. Since $f_{n}\to f$ uniformly $\exists{N_1} \text{ s.t. } \left|f(t)-f_{n}(t)\right| <\frac{\epsilon}{3}$ if $n\ge N_1$ for all $t \in E$.\\
			Since $A_{n}\to A$, $\exists{N_2} \text{ s.t. } \left|A_{n}-A\right| < \frac{\epsilon}{3}$ if $n\ge N_2$.
			\\
			Let $N=\max\{N_1,N_2\}$ and use $n=N$ in (*).
			Then
			\[
				\forall{t \in E}: \left|f(t)-A\right|\le  \frac{\epsilon}{3}+\left|f_N(t)-A_N\right| +\frac{\epsilon}{3}
				.\]
			Since $\lim_{t\to x}{f_N(t)}=A_N$, there exists $\delta>0$ s.t. $t \in N_{\delta}^{E}(x)\setminus \{x\}  \implies \left|f_N(t)-A_{N}\right| < \frac{\epsilon}{3}$.
			Then
			\[
				t \in N_{\delta}^{E}(x)\setminus \{x\}  \implies \left|f(t)-A\right|< \frac{\epsilon}{3}+\frac{\epsilon}{3}+\frac{\epsilon}{3}=\epsilon
			\]
			Hence, $\lim_{t\to x}{f(t)}=A$.
		\end{describe}
	\end{proof}
\end{thm}
\begin{Corollary}
	\label{cor:7.12}
	If $f_{n}$ is continuous on $E$ and $f_{n}\to f$ uniformly on $E$, then $f$ is continuous on $E$.
	\begin{proof}
		Every function is continuous at an isolated point, so only need to consider $x \in E' \cap E$.\\
		\begin{flalign*}
			f(x) & :=\lim_{n\to \infty}{f_{n}(x)}  = \lim_{n\to \infty}{\lim_{t\to x}{f_{n}(t)}} & (\because f_{n} \text{ continuous})      \\
			     & =\lim_{t\to x}{\lim_{n\to \infty}{f_{n}(t)}}=\lim_{t\to x}{f(t)}              & (\because \text{Theorem~\ref{thm:7.11}})
			.\end{flalign*}
	\end{proof}

	\begin{remark}
		VERY IMPORTANT
	\end{remark}
\end{Corollary}

\begin{thm}[13]
	Suppose $K$ is compact and
	\begin{enumerate}
		\item $f_n$ is continuous on $K$ for all $n \in \N$
		\item $f_{n}\to f$ pointwise on $K$ and $f$ is continuous on $K$
		\item $f_{n}(x)\ge f_{n+1}(x)$ $\forall x \in K$, $\forall n \in \N$.
	\end{enumerate}
	Then $f_{n}\to f$ uniformly on $K$.
	\begin{proof}
		Let $g_{n}=f_{n}-f$. Then
		\begin{enumerate}
			\item $g_{n}$ is continuous on $K$ for all $n \in \N$
			\item $g_{n}\to 0$ pointwise on $K$
			\item $g_{n}(x)\ge g_{n+1}(x)$ $\forall x \in K$, $\forall n \in \N$.
		\end{enumerate}
		Goal: Prove $g_{n}\to 0$ uniformly on $K$; i.e., given $\epsilon>0$, $\exists{N}$ s.t. $\forall{n\ge N}: \forall{x \in K}: \left|g_{n}(x)\right|<\epsilon$.\\
		For this, it suffices if $\exists{N} \text{ s.t. } g_N(x)<\epsilon$ for all $x \in K$.\\
		Let $K_n=g_n^{-1}([\epsilon,\infty])$.\\
		Then the goal becomes: find $N$ s.t. $K_{N}=\emptyset$, since this implies $K=K_N^{c}=g_{N}^{-1}([0,\epsilon])$.\\
		Since $g_{n}$ is continuous, $K_{n}$ is closed and since $K_{n} \subset K$, $K_{n}$ is compact. Also $K_{n+1} \subset K_{n}$ because $g_{n+1}(x)\ge \epsilon \implies g_{n}(x)\ge \epsilon$.\\
		Let $x \in K$. Since $g_{n}(x)\to 0$, $\exists{N_{x}} \text{ s.t. } \underbrace{x \not\in K_{N_x}}_{\because g_{n}(x)<\epsilon \text{ for large } n}$ for all $n \ge N_x$.
		Hence, $x \not\in \bigcap_{n=1}^{\infty}K_n$.
		$x$ is arbitrary, so $\bigcup_{n=1}^{\infty}K_n=\emptyset$.
		By corollary to Theorem~\ref{thm:2.36}, this implies $\exists{N } \text{ s.t. } K_N=\emptyset$.
	\end{proof}
\end{thm}
\begin{example}
	Let \[
		f_{n}(x)= \begin{cases}
			1-nx & 0<x\le \frac{1}{n} \\
			0    & \frac{1}{n}<x\le 1
		\end{cases}
		.\]
	\begin{enumerate}
		\item Let $K=(0,1]$ (not compact).
		      $f_{n}(x)\to 0$ for all $x \in K$ so (a), (b), (c) all hold, but $f_{n}$ does not converge uniformly to zero function on $K$ as $K$ not compact.
		      \begin{center}
			      \begin{tikzpicture}
				      \begin{axis}[
						      axis lines = left,
						      xlabel = $x$,
						      ylabel = {$f_n(x)$},
						      ymin=0,ymax=1.5,
						      xmin=0,xmax=1.5,
					      ]
					      \addplot [
						      domain=0:1,
						      samples=100,
						      color=blue,
					      ]
					      {1-x};
					      \addlegendentry{$f_n(x)=1-nx$}
					      \addplot [
						      domain=0:1,
						      samples=100,
						      color=red,
					      ]
					      {0};
					      \addlegendentry{$f_n(x)=0$}
				      \end{axis}
			      \end{tikzpicture}
		      \end{center}
		\item Let $K=[0,1]$ (compact).
		      $f_{n}$ continuous, $f_{n}(x)\to 0$ for all $x \in K$ but not uniformly. Here, (a) and (b) hold, but (c) does not.
	\end{enumerate}
\end{example}
\begin{define}[14]
	For a metric space $X$,
	let \[
		\mathscr{C}(X)=\{f: X \to \C \text{ s.t. } f \text{ is continuous}\}
		.\]
	The supremum norm of $f \in \mathscr{C}(X)$ is defined by \[
		\|f\|=\sup_{x \in X}\{\left|f(x)\right|\}
		.\]
	\begin{notation}
		When $X=[a,b]$, often write $\|f\|_{\infty}$ instead of $\|f\|$ since $\lim_{p\to \infty}{\|f\|_p}= \|f\|_{\infty}$ (C.f A3-Q2).
	\end{notation}
\end{define}

\begin{prop}
	\[
		d(f,g)= \|f-g\| \text{ defines a metric on } \mathscr{C}(X)
		.\]
	Proof:
	\begin{enumerate}
		\item {$d(f,g)=0 \iff \sup_{x \in X}\{\left|f(x)-g(x)\right| \}=0$ and hence $f(x)=g(x)$  $\forall x \in X$; i.e., $f=g$.}
		\item {$d(f,g)=d(g,f)$}
		\item {$d(f,g)\le d(f,h)+d(h,g)$ since $\forall{x \in X}: \left|f(x)-g(x)\right|\le \left|f(x)-h(x)\right|+ \left|h(x)-g(x)\right|\le \|f-h\|+ \|h-g\|$. Hence, $\|f-g\|\le \|f-h\|+\|h-g\|$.}
	\end{enumerate}
	As a consequence, $f_{n}\to f$ uniformly on $X$ if and only if $f_{n}\to f$ in the metric space $(\mathscr{C}(X),\|\cdot\|)$.
	Proof:	\begin{flalign*}
		\text{LHS} & \Leftrightarrow  \forall{\epsilon > 0}: \exists{N} \text{ s.t. } \left|f_{n}(x)-f(x)\right|<\epsilon  \text{ if $n\ge N$ for all $x \in X$}. \\
		           & \Leftrightarrow \forall{\epsilon > 0}: \exists N \text{ s.t. }  \|f_{n}-f\|<\epsilon \text{ if } n\ge N                                      \\
		           & \Leftrightarrow f_{n}\to f \text{ in } (\mathscr{C}(X),\|\cdot\|)
		.\end{flalign*}
\end{prop}

\begin{thm}[15]
	$\mathscr{C}$ is a complete metric space.
	\begin{proof}
		Let $\{f_{n}\} $ be a Cauchy sequence in $\mathscr{C}(X)$; i.e., \[
			\forall{\epsilon > 0}: \exists N \text{ s.t. } n,m\ge N \implies \|f_{m}-f_{n}\| <\epsilon
			.\]
		By Cauchy criterion (Theorem~\ref{thm:7.8}), $f_{n}\to f$ uniformly for some $f$.\\
		Now it is sufficient to check that $f \in \mathscr{C}(X)$.\\
		By Cor~\ref{cor:7.12}, $f$ is continuous.
		Also, $f$ is bounded since $\exists{N_0} \text{ s.t. } \left|f(x)-f_{N_0}(x)\right| <1$ for all $x \in X$.
		Then \[
			\left|f(x)\right| \le \left|f_{N_0}(x)\right| +\left|f(x)-f_{N_0}(x)\right| \le \underbrace{M_0+1}_{\text{bound for $f_{N_0}$}} \text{ for all } x \in X
			.\]
		$\therefore f \in \mathscr{C}(X)$.
	\end{proof}
\end{thm}

\begin{thm}[16]
	Suppose $f_{n} \in \mathscr{R}_{\alpha}[a,b]$ for $n \in \N$ and $f_{n}\to f$ uniformly on $[a,b]$.
	Then $f \in \mathscr{R}_{\alpha}[a,b]$ and \[
		\int_{a}^{b}{f\mathrm{d}\alpha}=\lim_{n\to \infty}{\int_{a}^{b}{f_{n}\mathrm{d}\alpha}}
		.\]
\end{thm}


