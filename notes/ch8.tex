\chapter{8}
Power series, $e^{x}, \log{x},\sin{x},\cos{x}$, Fourier series (we're omitting Gamma function)


\section{Power Series}
Recall $\sum_{n=0}^{\infty}{c_n{x^{n}}}$ has a radius of convergence $R=\frac{1}{\limsup_{n\to \infty}{\sqrt[n]{\left|c_{n}\right| }}}$ such that the series absolutely converges for $\left|x\right| < R$, diverges for $\left|x\right| >R$ and anything possible for $\left|x\right| =R$.

\begin{remark}
	To determine $R$, we often use the ratio test instead:
	\[
		\left|\frac{c_{n+1}x^{n+1}}{c_nx^{n}}\right| =\left|x\right| \cdot \left|\frac{c_{n+1}}{c_n}\right|\underbrace{\to}_{\text{ if limit exists } } \left|x\right| \cdot L\implies \text{ absolute convergence if } \left|x\right| <\frac{1}{L}, \text{ diverges if }  \left|x\right| >\frac{1}{L}
		,\] where $R=\frac{1}{L}=\frac{1}{\lim_{n\to \infty}{\left|\frac{c_{n+1}}{c_n}\right| }}$.\\
	In general, by Theorem~\ref{thm:3.37}, we also have \[
		\frac{1}{\lim_{n\to \infty}{\left|\frac{c_{n+1}}{c_n}\right|}}\le R\le \frac{1}{\liminf_{n\to \infty}{\frac{c_{n+1}}{c_n}}}
		.\]
\end{remark}

\begin{theorem}[1]
	Suppose $\sum_{n=0}^{\infty}{c_n x^{n}}$ has a radius of convergence $R>0$ and define $f(x)=\sum_{n=0}^{\infty}{c_n x^{n}}$ for $\left|x\right| < R$. Such a function $f(x)$ is called an \textit{analytic function} .\\
	If $R<\infty$, then the series converges uniformly on $[-R+\epsilon,R-\epsilon]$ for all $\epsilon>0$.\\
	If $R=\infty$, then the series converges uniformly on $[-M,M]$ for all $M < \infty$.
	The function $f$ is continuous and differentiable on $(-R,R)$ with $f'(x)=\sum_{c_n n x^{n-1}}$
	\begin{note}
		Uniform convergence may not hold on $(-R,R)$. C.f. A7-Q3.
	\end{note}
	\begin{proof}
		\hfill
		\begin{description}
			\item[Uniform convergence:]
			      For $\left|x\right| \le R-\epsilon$,
			      if $\left|c_n x^{n}\right|\le \left|c_n\right| (R-\epsilon)^{n}$.\\
			      Since $\sum_{n=0}^{\infty}{\left|c_n\right|(R-\epsilon)^{n}}<\infty$ by absolute convergence on $(-R,R)$. Hence, by Weierstrass M-test, $\sum_{n=0}^{\infty}{c_n x^{n}}$ converges uniformly on $[-R+\epsilon,R-\epsilon]$.\\
			\item[Derivative:]
			      The radius of convergence of $\sum_{n=1}^{\infty}{nc_n x^{n-1}}$ is \[
				      \frac{1}{\limsup_{n\to \infty}{\sqrt[n]{n \left|c_n\right| }}}=\frac{1}{\limsup_{n\to \infty}{\sqrt[n]{n}\sqrt[n]{\left|c_n\right|}}}=\text{ radius of convergence of } \sum_{n=\infty}^{c_n}{x^{n}}
				      .\]
			      (Note that $\lim_{n\to \infty}{\sqrt[n]{n}}=1$.)\\
			      Let $S_n(x)=\sum_{m=0}^{n}{c_m x^{m}}$. Then $S'_n(x)=\sum_{m=1}^{n}{c_m m x^{m-1}}$.
			      By the first part of the proof, $S'_n(x)\to \sum_{m=1}^{\infty}{c_m m x^{m-1}}$ uniformly on $[-R+\epsilon, R-\epsilon]$. Since also $S_n(x)\to f(x)$, by Theorem~\ref{thm:7.17}, $f'$ exists on $[-R+\epsilon,R-\epsilon]$, and $f'(x)=\sum_{m=1}^{\infty}{c_m m x^{m-1}}$.\\
			      Since $\epsilon$ is arbitrary, $f'(x)=\sum_{m=1}^{\infty}{c_m m x^{m-1}}$ for all $x \in (-R,R)$. In particular, $f$ is also continuous.
		\end{description}
	\end{proof}
\end{theorem}

\begin{corollary}
	If $f(x)=\sum_{n=0}^{\infty}{c_n x^{n}}$ converges for $\left|x\right| < R$, then $f^{(k)}(x)$ exists for all $k \in \N$, and
	\begin{equation*}
		f^{(k)}(x)=\sum_{n=k}^{\infty}{c_n n(n-1)\cdots (n-k+1)x^{n-k}}
		\tag{*}
		.\end{equation*}
	Consequently, $c_k=f^{(k)}(0)$ and $f(x)=\sum_{n=0}^{\infty}{\frac{f^{(n)}(0)}{n!} x^{n}}$.
	\begin{note}
		C.f. Taylor's theorem.
	\end{note}
	\begin{proof}
		By Theorem~\ref{thm:8.1}, $f'(x)=\sum_{n=1}^{\infty}{c_n n x^{n-1}}, f''(x)=(f')'(x)=\sum_{n=2}^{\infty}{c_n n(n-1)x^{n-2}}, \ldots $.\\
		Set $x=0$ in (*) to get $f^{(k)}(0)\underbrace{=}_{\text{ only } n=k \text{ term survives } }c_k k (k-1) \cdots \cdot 1=c_k k!$.
	\end{proof}
\end{corollary}
\begin{example}
	Let $f(x)=\begin{cases}
			e^{-\frac{1}{x^2}} & x\neq 0 \\
			0                  & x=0
		\end{cases}$.
	By Rudin's problem 8.1, $f^{(n)}(0)=0$ for all $n=0,1,2,\ldots $, so
	$f(x)\neq \sum_{n=0}^{\infty}{\frac{f^{(n)}}{n!} x^{n}}$ except for $x=0$.
	\begin{remark}[Bump Function]
		\textit{Bump functions} are infinitely differentiable functions with compact support. E.g.,
		\[
			f(x)=\begin{cases}
				e^{-\frac{1}{1-x^2}} & x \in (-1,1)         \\
				0                    & \left|x\right| \ge 1
			\end{cases}
			.\]
	\end{remark}
\end{example}

\begin{thm}[2][\namedlabel{thm:abel}{Abel's Theorem}]
	Suppose $\sum_{n=0}^{\infty}{c_n}$ converges (perhaps conditionally). Let $f(x)=\sum_{n=0}^{\infty}{c_n x^{n}}$.
	Then $f(x)$ converges for $\left|x\right| < 1$ and $\lim_{x\to 1^{-}}{f(x)}=f(1)=\sum_{n=0}^{\infty}{c_n}$.
	\begin{remark}
		Interesting case is at $R=1$, since $R>1$ implies continuity at of $f$ for $\left|x\right| < R$.
	\end{remark}
	\begin{proof}
		By the root test, $\limsup_{n\to \infty}{\sqrt[n]{\left|c_n\right|}}\le 1$, so $\sum_{n=0}^{\infty}{c_n x^{n}}$ has $R\ge 1$.
		Let $S_n=\sum_{m=0}^{n}{c_m}$ and $S=\sum_{m=0}^{\infty}{c_m}=\lim_{n\to \infty}{S_n}$.\\
		Set $S_{-1}=0$. Then $c_{n}=s_{n}-s_{n-1}$ for $n\ge 0$.\\
		Let $\epsilon>0$. We need to show $\exists{\delta > 0} \text{ such that } 1-\delta<x<1 \implies \left|f(x)-S\right| < \epsilon$.\\
		Start with partial sum for $f(x)$. For $\left|x\right| < 1$,
		\begin{flalign*}
			\sum_{m=0}^{n}{c_m x^{m}}=\sum_{m=0}^{n}{(S_m-S_{m-1})x^{m}}=\sum_{m=0}^{n}{S_m x^{m}}-\sum_{m=0}^{n}{S_{m-1}x^{m}}
			.\end{flalign*}
		Let $k=m-1$ so that $m=k+1$. Then
		\begin{flalign*}
			\sum_{m=-1}^{n}{S_{m-1}x^{m}} & =x\sum_{k=0}^{n-1}{S_k x^{k}}                                                                                                                     \\
			\sum_{m=0}^{n}{c_m x^{m}}     & =(1-x)\sum_{m=0}^{n}{S_m x^{m}}+\underbrace{S_n x^{n+1}}_{\to 0 \text{ as } n\to \infty \text{ since $S_n$ is bounded and $\left|x\right| < 1$} }
			.\end{flalign*}
		Let $n\to \infty$.
		Then
		\begin{flalign*}
			f(x) & =(1-x)\sum_{n=0}^{\infty}{S_n x^{n}}+0=(1-x)S
			.\end{flalign*}
		\begin{flalign*}
			\left|f(x)-S\right| & =\left|(1-x)\sum_{n=0}^{\infty}{S_n x^{n}}-S(1-x)\sum_{n=0}^{\infty}{x^{n}}\right| \\                                                                                   & =(1-x) \left|\sum_{n=0}^{\infty}{(S_n-S)x^{n}}\right|\le (1-x)\sum_{n=0}^{\infty}{\left|S_n-S\right|\cdot \left|x \right|^{n}  }
			.\end{flalign*}
		Choose $N$ s.t. $n\ge N \implies \left|S_n-S\right|<\frac{\epsilon}{2} $.
		For $x \in (0,1)$,
		\[
			\left|f(x)-S\right| \le (1-x) \sum_{n=0}^{N}{\left|S_n-S\right| x^{n}}+(1-x) \sum_{n=N+1}^{\infty}{\left|S_n-S\right|x^{n}}<\frac{\epsilon}{2}\cdot (1-x)\cdot \frac{1}{1-x}
			.\]
		$(1-x)\sum_{n=0}^{N}{\left|S_n-S\right| x^{n}}$ is a polynomial in $x$, so continuous and equals $0$ at $x=1$, so $<\frac{\epsilon}{2}$ if $\left|x-1\right| <\delta$ for some $delta>0$.
		Therefore, for $1-\delta<x<1$, $\left|f(x)-S\right|<\frac{\epsilon}{2}+\frac{\epsilon}{2}=\epsilon$.
	\end{proof}
	\begin{note}
		For an application of Abel's theorem, see Rudin's p. 175.\\
		For the case $\sum_{n=0}^{\infty}{c_n}=\infty$, see A7.
	\end{note}
\end{thm}

\begin{thm}[3]
	If $\sum_{i=1}^{\infty}{\sum_{j=1}^{\infty}{\left|a_{ij}\right|}}$, then \[
		\sum_{i=1}^{\infty}{\sum_{j=1}^{\infty}{a_{ij}}}= \sum_{j=1}^{\infty}{\sum_{i=1}^{\infty}{a_{ij}}},\] where both sides converge.
	\begin{proof}
		Rudin has a too clever proof \ldots  A7 involves a more straightforward proof.
	\end{proof}
\end{thm}

\begin{thm}[4]
	Suppose $f(x)=\sum_{n=0}^{\infty}{c_n x^{n}}$ (Taylor series of $f$ at $x=0$, a.k.a Maclauren series) has a radius of convergence $R>0$.
	Let $\left|a\right|<R$.
	Then $f(x)=\sum_{n=0}^{\infty}{\frac{f^{(n)}(a)}{n!}(x-a)^{n}}$ for (at least) $\left|x-a\right|<R-\left|a\right|$.
	\begin{proof}
		Note \[
			f(x)=\sum_{n=0}^{\infty}{c_n \left( a+(x-a) \right)^{n}}=\sum_{n=0}^{\infty}{c_n}\sum_{m=0}^{n}{\binom{n}{m}(x-a)^{m}a^{n-m}}
			.\]
		We want to interchange the order of summation.\\
		By Theorem~\ref{thm:8.3}, interchange of summations is justified if \[
			\sum_{n=0}^{\infty}{\sum_{m=0}^{n}{\left|c_n\right|\binom{n}{m}\left|x-a\right|^{m}\left|a\right|^{n-m}}}<\infty
			.\]
		Since $\text{LHS} =\sum_{n=0}^{\infty}{\left|c_{n}\right|\left( \left|x-a\right|+\left|a\right|\right)^{n}}$, which does converge if $\left|x-a\right|+\left|a\right|<R$, which we are assuming.\\
		Therefore, \[
			f(x)=\sum_{m}^{\infty}{\sum_{n}^{\infty}{c_n \binom{n}{m}a^{n-m}(x-a)^{m}}}=\sum_{m}{\left(\sum_{n}{c_n \binom{n}{m}a^{n-m}}\right)(x-a)^{m}}
			.\]
	\end{proof}
\end{thm}

